This paper will describe the implementation of Simultaneous Localization and Mapping using a FastSLAM algorithm, which implements a particle filter-based approach. This method allows the UAV to estimate its location while also building a map of the environment at the same time, a crucial aspect of autonomous navigation. This method allows the UAV to be put into any unknown environment and operate in a stable and controlled manner while providing a map of the environment it is in. The algorithm utilized motion estimation, landmark detection, the use of a Kalman filter to update state estimations, and resampling of particles based on their weights. The paper will explain the steps that are undertaken to implement a FastSLAM algorithm, including the theory for motion prediction, the steps required for data association using the Mahalanobis distance, how to mitigate/account for noise in sensor measurements and motion, and ensuring the most accurate estimate of your states while implementing the Kalman filter. It will further highlight the importance of using constraints for angular and distance measurements to ensure a consistent and reliable system without facing any numerical instability issues. The nomenclature used in the report can be found in section 2, and the implementation of FastSLAM can be shown to be broken down in section 3.