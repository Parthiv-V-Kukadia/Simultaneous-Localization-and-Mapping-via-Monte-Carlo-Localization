This report explored implementing SLAM using a FastSLAM algorithm by using a particle filter to estimate the trajectory and build a map of the environment The particle filter was used to mitigate noisy sensor data and support a non-linear motion model, mimicking real-world UAV implementation of SLAM. FastSLAM was utilized to allow the UAV to fine-tune its position estimates while developing a map of its surroundings in unknown environments through using the theory of motion prediction, data association, and resampling of particles.\\

The code implemented focused on particle resampling based on their weights, using the Kalman filter to update landmark positions, and matching corners for motion estimation. The entire algorithm was developed using a constraint that normalized angular measurements to ensure that heading and angle measurements were consistent and numerical instability was avoided. The implementation of FastSLAM showed how a UAV can localize and map simultaneously while performing robustly and maintaining high computational speeds. FastSLAM is shown to be a strong tool to use when tackling autonomous navigation problems. The system can be further improved by focusing on better data association techniques and implementing an optimized particle filtering process to allow localization in unknowns and mapping of more complex environments.